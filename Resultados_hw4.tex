\documentclass{article}
\usepackage[utf8]{inputenc}

\title{Resultados hw4}
\author{Diego Moreno 201513349 }
\date{Noviembre 2018}

\usepackage{natbib}
\usepackage{graphicx}

\begin{document}

\maketitle

\section{Parte 1  }
en esta parte donde basicamente: se tiene el valor inicial de una variable X(t)
y la formula de la derivada dx/dt. Se quiere conocer la variable X entre tiempos dados.

El nuevo punto X1 se calcula como
X1 = X0 + dt*pendiente

En lepafrog, la pendiente es la velocidad medio instante de tiempo despues. 
En RK4 la pendiente es un promedio ponderado entre las velocidades que se pueden dar entre t0 y t1.

\section{ Proyectil con friccion}
Las graficas no son parabolas porque la friccion frena los proyectiles. Las caidas del final son mas pronunciadas que en la
